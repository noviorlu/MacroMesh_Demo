\documentclass {article}
\usepackage{fullpage}
\usepackage[backend=bibtex]{biblatex}
\addbibresource{refereneces.bib}
\begin{document}

~\vfill
\begin{center}
\Large

A5 Project Proposal

Title: Real-time ClusterLOD Renderer

Name: Yang Chen

Student ID: 20816397

User ID: y2588che
\end{center}
\vfill ~\vfill~
\newpage
\section{Purpose}
	To tie together three totally unrelated rendering issues.

\section{Statement}
	For Ray Tracers: Paragraph describing interesting scene to be
		rendered and what features are needed to achieve
		this scene.

	Paragraph: What it's about.

	Paragraph: What to do.

	Paragraph: Why it is interesting and challenging.

	Paragraph: What I will learn

\section{Technical Outline}
     This project aims to enhance mesh rendering through efficient Level of Detail (LOD) management, starting with partitioning 
     the mesh into spatial clusters using METIS and simplifying them with Quadric Error Metrics (QEM) to form a multi-level LOD 
     hierarchy. A Bounding Volume Hierarchy (BVH) will then organize these LODs, supporting fast, compute shader-based selection 
     and streaming of visible clusters. A visibility buffer will store per-pixel geometry data, enabling deferred shading, while 
     additional effects like Screen Space Reflection (SSR), Ambient Occlusion (SSAO), and Multi-Sample Anti-Aliasing (MSAA) will 
     improve realism without excessive computational cost. Post-processing effects (bloom, tone mapping, and color grading) and 
     the Disney Principled BSDF shader for realistic material rendering will further enhance visual quality, all while utilizing 
     an indexed mesh structure to reduce memory usage.

\section{Bibliography}
     Articles and/or books with important information on the topics of the project are referenced below.
     
\begin{itemize}
     \item Visibility Buffer techniques are explored in \cite{FilmicWorlds}.
     \item Nanite technology for massive model visualization is detailed in \cite{Karis2021Nanite}.
     \item Deferred Shading methods are discussed in \cite{LearnOpenGLDeferred}.
     \item Screen Space Ambient Occlusion (SSAO) is explained in \cite{LearnOpenGLSSAO}.
\end{itemize}

\newpage


\noindent{\Large\bf Objectives:}
\begin{enumerate}
     \item[\_\_\_ 1:]  [PRECOMPUTE-GEO] Mesh Partition
     \\ Use the METIS library to partition the triangle mesh into multiple clusters, each representing a group of spatially 
     adjacent triangles. This partitioning process defines the 0-LOD (highest detail) mesh. To optimize for multi-level 
     rendering, select {\bf N} adjacent clusters and combine them into larger entities known as Cluster Groups, which can 
     later be used to construct higher levels of detail.
 
     \item[\_\_\_ 2:]  [PRECOMPUTE-GEO] QEM Simplification
     \\ Apply the Quadric Error Metric (QEM) simplification algorithm to the previously created Cluster Groups. This 
     simplification process reduces the number of triangles in each cluster while preserving the visual fidelity as much as 
     possible. Track the simplification error for each cluster and Cluster Group to guide dynamic LOD selection later. This 
     error metric will be critical for runtime decisions on which clusters to render at different distances.
 
     \item[\_\_\_ 3:]  [PRECOMPUTE-GEO] LOD Error Tree Management
     \\ Construct a Level of Detail (LOD) hierarchy by recursively applying the mesh partitioning and QEM simplification 
     steps (Objectives 1 and 2) to generate progressively lower levels of detail. During this process, ensure that the 
     simplification error grows monotonically with each LOD level. This hierarchical structure will allow for smooth 
     transitions between LODs at runtime.
 
     \item[\_\_\_ 4:]  [PRECOMPUTE-GEO] LOD-BVH Construction
     \\ Create a Bounding Volume Hierarchy (BVH) for each LOD level using a Surface Area Heuristic (SAH) approach. In this 
     BVH, each node represents a Cluster Group, which allows for efficient spatial queries. This BVH structure will support 
     fast LOD selection and culling during rendering by organizing Cluster Groups spatially and hierarchically.
 
     \item[\_\_\_ 5:]  [RUNTIME-GEO] LOD Selection \& Streaming
     \\ At runtime, implement LOD selection and streaming using a Compute Shader. A job scheduler system traverses the 
     LOD-BVH, identifying which Cluster Groups are within the view frustum and within a desired error threshold. Worker 
     threads then go through clusters in each selected Cluster Group to further refine visibility based on error metrics. 
     This hierarchical approach enables selective loading of clusters based on viewer distance and screen resolution.
 
     \item[\_\_\_ 6:]  [RUNTIME-GEO] Visibility Buffer
     \\ Implement a visibility buffer to record per-pixel information about the scene. For each pixel, store the barycentric 
     coordinates of the triangle it falls within, along with the depth information. This buffer will be essential for deferred 
     shading, as it enables deferred rendering without storing full geometry data per frame.
 
     \item[\_\_\_ 7:]  [RUNTIME-PIPELINE] Deferred Shading
     \\ Utilize the visibility buffer to perform deferred shading, allowing for efficient lighting and shading calculations. 
     Instead of recalculating light interactions per triangle, deferred shading operates directly on the per-pixel visibility 
     data, reducing the computational load and enhancing performance in scenes with multiple light sources.
 
     \item[\_\_\_ 8:]  [RUNTIME-SHADING] Screen Space Reflection (SSR) and Ambient Occlusion (SSAO)
     \\ Implement SSR and SSAO effects using the visibility buffer. Screen Space Reflection adds realistic reflections by tracing 
     rays in screen space, while Ambient Occlusion provides subtle shading in occluded areas. These effects add realism to the 
     scene without requiring ray tracing, balancing visual fidelity with performance.
 
     \item[\_\_\_ 9:]  [RUNTIME-SHADING] Post-Processing Effects
     \\ Apply a range of post-processing effects, including bloom, tone mapping, and color grading, to the final image. These 
     effects use the visibility buffer to add cinematic visual quality and enhance color and brightness dynamics in the rendered scene.
 
     \item[\_\_\_ 10:]  [PRECOMPUTE-OPTIM] Index Buffer Storage Utilization
     \\ Modify the mesh structure to use indexed vertices, which reduces memory usage by reusing shared vertices across triangles. 
     By switching to an indexed format, the overall data footprint for vertex storage is minimized, especially in large meshes with 
     numerous shared edges.
 
     \item[\_\_\_ 11:]  [ADDI RUNTIME-OPTIM] Multi-Sample Anti-Aliasing (MSAA)
     \\ Implement MSAA to improve visual quality by reducing jagged edges in rendered images. Enable MSAA before creating the 
     visibility buffer to ensure that anti-aliasing is applied during shading. This approach smooths out edges and contributes to 
     the overall visual appeal of the rendering.
 
     \item[\_\_\_ 12:]  [ADDI RUNTIME-SHADING] Disney Principled BSDF Shader
     \\ Integrate the Disney Principled BSDF shader model to render materials more realistically. This shader model, developed by 
     Disney, offers a unified approach to rendering different material types, balancing realism with performance, and simplifying 
     shader complexity. It allows for smoother transitions between material properties and supports a wide variety of looks.
 
\end{enumerate}
 
\printbibliography

% Delete % at start of next line if this is a ray tracing project
% A4 extra objective:
\end{document}